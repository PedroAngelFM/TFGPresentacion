\subsection{Análisis de componentes principales}
\begin{frame}{Métodos no supervisados}
\begin{columns}
\begin{column}{0.9\textwidth}
Dado un vector aleatorio $\mathbf{x}$ de longitud $p$. Sea $r=rg(\mathbf{X})$. Entonces se definen las componentes principales:
\begin{defi}
Se definen las componentes principales 
\begin{equation}
Z_j=v_{1j}X_1+\ldots v_{pj}X_p=\mathbf{v}_j^T\mathbf{x}^T \quad j=1\ldots r
\end{equation}

\noindent Donde $\textbf{v}_j$ es un vector columna con $p$ escalares y la nueva variable aleatoria $Z_j$ cumple lo siguiente:
\begin{itemize}
\item Si $j=1$ $Var(Z_1)$ es máxima restringido a $\mathbf{v}_1^T \mathbf{v}_1=1$
\item Si $j>1$ debe cumplir:
\begin{itemize}
\item $Cov(Z_j,Z_i)=0\quad \forall i\neq j $
\item $\textbf{v}_j^T \textbf{v}_j=1$
\item $Var(Z_j)$ es máxima. 
\end{itemize}
\end{itemize}
\end{defi}
\end{column}
\end{columns}
\end{frame}


\begin{frame}{Métodos no supervisados}
\begin{columns}
\begin{column}{0.9\textwidth}
\begin{defi}
Dada una matriz $\textbf{X}\in  \mathbb{M}_{N\times p}(\mathbb{R})$ existe la descomposición en valores singulares :
\begin{equation}
\textbf{X}=\textbf{U}\mathbf{D}\textbf{V}^T
\end{equation}
Donde: 
\begin{itemize}
\item \textbf{U} matriz ortogonal y de tamaño $N \times r$
\item $\mathbf{D}$ matriz de tamaño $r \times r$ diagonal, cuyos elementos no nulos son los valores singulares $\sigma_1\geq\ldots\geq \sigma_r\geq 0$, es decir
\begin{equation}
\mathbf{D}=\begin{pmatrix}
\sigma_1 & 0 & \cdots  & \cdots \\
0 & \sigma_2 & 0 & \cdots\\
\vdots & \vdots & \vdots & \vdots\\
\cdots & \cdots & 0 & \sigma_r
\end{pmatrix}
\end{equation}
\item \textbf{V} matriz ortogonal y de tamaño $p \times r$. 
\end{itemize}
\end{defi}
\end{column}
\end{columns}
\end{frame}

\begin{frame}{Métodos no supervisados}
\begin{columns}
\begin{column}{0.9\textwidth}

Las componentes principales se pueden calcular de dos maneras:
\begin{itemize}
\item Descomposición en vectores y valores propios de la matriz de covarianzas. 
\item Descomposición en valores singulares de la matriz de datos

\end{itemize}
\end{column}
\end{columns}
\end{frame}




\begin{frame}{Métodos no supervisados}
\begin{columns}
\begin{column}{0.9\textwidth}
\begin{defi}
Se llama matriz reducida de orden $m\leq r$ de $\textbf{X}$ y se denota como $\textbf{X}_m$, a la matriz $N\times m $ resultado de:
\begin{equation}
\textbf{X}_m=\textbf{U}_m\mathbf{D}_m\textbf{V}^T_m
\end{equation}
Donde:
\begin{itemize}
\item $\textbf{U}_m$ matriz ortogonal de tamaño $N \times m$, resultado de tomar de \textbf{U} únicamente la matriz las $m$ primeras columnas. 
\item $\mathbf{D}_m$  matriz cuadrada de tamaño $m$ diagonal con los $m$ primeros valores singulares. 
\item $\textbf{V}_m$ matriz ortogonal de tamaño $p \times m$ obtenida al tomar las $m$ primeras columnas de \textbf{V}.
\end{itemize}
\end{defi}
\end{column}
\end{columns}
\end{frame}

\begin{frame}{Métodos no supervisados}
\begin{columns}
\begin{column}{0.9\textwidth}

\begin{defi}
Sea $\textbf{A}\in \mathbb{M}_{N\times p}(\mathbb{R})$. Definimos la norma de Frobenius de la matriz \textbf{A} como:
\begin{equation}
||\textbf{A}||_F=(tr(\textbf{A}^T \textbf{A}))^{\frac{1}{2}}=\left(\sum_{i=1}^{N}\sum _{j=1}^{p}a_{ij}^2\right)^{\frac{1}{2}}
\end{equation}
\end{defi}
\end{column}
\end{columns}
\end{frame}


\begin{frame}{Métodos no supervisados}
\begin{columns}
\begin{column}{0.9\textwidth}
\begin{teorema}[De Eckart-Young]
Sea \textbf{A} una matriz de coeficientes reales de tamaño $N\times p$ y rango $r$. Entonces se cumple que :
\begin{equation}
||\textbf{A}-\textbf{B}||_F\leq ||\textbf{A}-\textbf{A}_m||_F \quad \forall \textbf{B}/ rg(\textbf{B})=m \leq r
\end{equation} 
\end{teorema}

El número $m$ de componentes principales se puede elegir teniendo en cuenta la siguiente cantidad:
\begin{equation}
t_m=\dfrac{\sum_{j=1}^{m}\lambda_j} {\sum_{j=1}^{r}\lambda_j}
\end{equation}

La proporción de variabilidad acumulada por las $m$ primeras componentes.

\end{column}
\end{columns}
\end{frame}