\subsection{Clasificación}
\begin{frame}{Métodos supervisados}
\begin{columns}
\begin{column}{0.9\textwidth}
La clasificación es el caso donde la variable respuesta es cualitativa. 
Hay dos tareas a realizar:
\begin{itemize}
 \item Discriminación: Estudiar las características que diferencian a cada una de las poblaciones. 
 \item Clasificación: Asignar nuevas observaciones a una de las poblaciones 
\end{itemize}
\end{column}
\end{columns}
\end{frame}


\begin{frame}{Métodos supervisados}
\begin{columns}
\begin{column}{0.9\textwidth}
Si tenemos dos poblaciones $\pi_1, \pi_2$ de las cuales se conocen las funciones de probabilidad de cada una de ellas $f_1,f_2$ , entonces si se conoce la probabilidad de de clasificación de clasificar en la población incorrecta, $P(1|2), P(2|1)$, entonces: 
\begin{equation}
\begin{split}
P(i|\textbf{x}_0)&=\dfrac{P(\textbf{x}_0|i)P(i)}{P(1)P(\textbf{x}_0|1)+P(2)P(\textbf{x}_0|2)}\\&=\dfrac{f_i(\mathbf{x}_0)P(i)}{P(1)f_1(\mathbf{x}_0)+P(2)f_2(\mathbf{x}_0)}
\end{split}
\end{equation}
donde $i=1,2$
\end{column}
\end{columns}
\end{frame}

\begin{frame}{Métodos supervisados}
\begin{columns}
\begin{column}{0.9\textwidth}
Se utilizará el concepto de función discriminante: 
\begin{defi}
Se llama función discriminante, $f_{d}$ aquella que:
\begin{equation}
f_{d}:\mathbf{\Omega}\longrightarrow \mathbb{R},
\end{equation}
donde $\mathbf{\Omega}$ es el espacio de observaciones posibles, definida de tal manera que si $f_{d}(\textbf{x}_0)>0\Rightarrow \textbf{x}_0\in \pi_i$ y en caso contrario $\textbf{x}_0\notin \pi_i$. 
\end{defi}
\end{column}
\end{columns}
\end{frame}

\begin{frame}{Métodos supervisados}
\begin{columns}
\begin{column}{0.9\textwidth}
Teniendo esta definición en mente para dos poblaciones y teniendo  en cuenta el coste de clasificación errónea, se obtiene la siguiente función discriminante:

\begin{equation}
f_d(\mathbf{x}_0)=\dfrac{f_1(\textbf{x}_0)P(1)}{c(1|2)}-\dfrac{f_2(\textbf{x}_0)P(2)}{c(2|1)}
\end{equation}

Si se sustituye $f_1,f_2$ por funciones normales se obtiene la función discriminante: 
\begin{equation}
\begin{split}
f_d(\mathbf{x})&=(\textbf{x}-\mu_1)^T \mathbf{\Sigma}^{-1}(\textbf{x}-\mu_1)+log\left(\dfrac{P(1)}{c(1|2)}\right)\\&-(\textbf{x}-\mu_2)^T \mathbf{\Sigma}^{-1} (\textbf{x}-\mu_2)-log\left(\dfrac{P(2)}{c(2|1)}\right)
\end{split}
\end{equation}

\end{column}
\end{columns}
\end{frame}

\begin{frame}{Métodos supervisados}
\begin{columns}
\begin{column}{0.9\textwidth}

El análisis canónico de poblaciones busca identificar las direcciones de mayor diferenciación entre las distintas $L$ poblaciones.
\begin{defi}
\noindent La covarianza intragrupo de un par de variables en la $l$-ésima población:
\begin{equation}
W_l(X_j,X_{j'})=\dfrac{1}{N_l}\sum_{i\in I_l} (x_{ij}-\overline{x}_{jl})(x_{ij'}-\overline{x}_{j'l})
\end{equation}   
\end{defi}
\begin{defi}
La covarianza intergrupos
\begin{equation}
B(X_j,X_{j'})=\sum_{l=1}^L\dfrac{N_l}{N}(\overline{x}_{jl}-\overline{x}_{j})(\overline{x}_{j'l}-\overline{x}_{j'})
\end{equation}   
\end{defi}

\end{column}
\end{columns}
\end{frame}

\begin{frame}{Métodos supervisados}
\begin{columns}
\begin{column}{0.9\textwidth}

Entonces se buscan combinaciones lineales que maximicen la varianza entre grupos respecto la varianza por tanto, se busca maximizar la función
\begin{equation}
f(\mathbf{a})=\dfrac{\mathbf{a}^T\mathbf{Ba}}{\mathbf{a}^T\mathbf{Ta}}
\end{equation}

Tomando la restricción $\mathbf{a}^T\mathbf{T}\mathbf{a}$ y aplicando los multiplicadores de Lagrange, se obtiene que el vector $\mathbf{a}$ es el vector propio con valor propio máximo $\lambda$ de la matriz $\mathbf{T}^{-1}\mathbf{B}$.

\end{column}
\end{columns}
\end{frame}