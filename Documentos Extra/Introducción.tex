\section{Introducción}
\begin{frame}{Introducción}
\begin{columns}
\begin{column}{0.9\textwidth}
\begin{defi}
Los métodos multivariantes son el conjunto de técnicas que buscan describir y extraer información de las relaciones entre múltiples variables medidas en una o varias muestras u observaciones. 
\end{defi}
\end{column}
\end{columns}

\end{frame}

\begin{frame}{Introducción}
\begin{columns}
\begin{column}{0.9\textwidth}
Los métodos multivariantes pueden clasificarse de la siguiente manera según el caso de estudio:
\begin{itemize}
\item \textbf{Métodos supervisados:} estudian la relación estocástica que hay entre dos conjuntos de variables, un conjunto de variables predictoras y otro de variables respuesta. 
\begin{itemize}
\item \textit{Regresión:} la o las  variables variables respuesta son continuas. 
\item \textit{Clasificación: }la o las variables respuesta son cualitativas. 
\end{itemize}
\item \textbf{Métodos no supervisados: }analizan la estructura y relaciones que hay entre las variables.
\end{itemize}
\end{column}
\end{columns}
\end{frame}

\begin{frame}{Introducción}
\begin{columns}
\begin{column}{0.9\textwidth}
\textbf{Objetivo: }\\
Describir las principales técnicas multivariantes y dar un ejemplo de aplicación sobre datos reales de las mismas.
\end{column}
\end{columns}
\end{frame}
