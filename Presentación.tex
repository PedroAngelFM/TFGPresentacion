\documentclass[11pt]{beamer}
\usetheme{Madrid}

\usepackage[spanish]{babel}
\usepackage{amsmath}
\usepackage{amsfonts}
\usepackage{amssymb}
\usepackage{graphicx}
\usepackage{enumerate}
\usepackage{fancyhdr}
\usepackage{fancybox}
\usepackage{lastpage}
\usepackage{color}
\usepackage{amsbsy}
\usepackage{amsthm}
\usepackage{listings}


\usepackage{multirow}
\usepackage{subfig}
\usepackage{textcomp}
\usepackage{makeidx}%glosario de términos

\numberwithin{equation}{section}
\theoremstyle{definition}
\newtheorem{defi}{Definición}[section]

\theoremstyle{definition}
\newtheorem{propo}{Proposición}[section]

\theoremstyle{definition}
\newtheorem{coro}{Corolario}[section]

\theoremstyle{definition}
\newtheorem{teorema}{Teorema}[section]
\setlength{\textwidth}{5in}

\usepackage{srcltx}%permite la busqueda inversa, si no viene por defecto en la version de LaTeX usada.

% lo siguiente solo usarlo cuando compilo directamente en pdflatex
% si lo uso compilando en latex el {\'\i}ndice no lo hace bien
\usepackage{hyperref}

\hypersetup{colorlinks=true, linkcolor=black, urlcolor=black, citecolor=black}
\hypersetup{bookmarksopen=false, bookmarksnumbered=true} \hypersetup{pdfstartview=FitH}

\author{Pedro Ángel Fraile Manzano}
\title{Revisión de métodos multivariantes supervisados y no supervisados}
\AtBeginSection[]
{
  \begin{frame}<beamer>{Contenidos}
    \tableofcontents[currentsection,currentsubsection]
  \end{frame}
}

\AtBeginSubsection[]
{
  \begin{frame}<beamer>{Contenidos}
    \tableofcontents[currentsection,currentsubsection]
  \end{frame}
}
\author{Pedro Ángel Fraile Manzano}
%\title{}
%\setbeamercovered{transparent} 
%\setbeamertemplate{navigation symbols}{} 
%\logo{} 
%\institute{} 
%\date{} 
%\subject{} 
\begin{document}

\begin{frame}
\titlepage
\end{frame}

\begin{frame}
\tableofcontents
\end{frame}

\section{Introducción}
\begin{frame}{Introducción}
\begin{columns}
\begin{column}{0.9\textwidth}
\begin{defi}
Los métodos multivariantes son el conjunto de técnicas que buscan describir y extraer información de las relaciones entre múltiples variables medidas en una o varias muestras u observaciones. 
\end{defi}
\end{column}
\end{columns}

\end{frame}

\begin{frame}{Introducción}
\begin{columns}
\begin{column}{0.9\textwidth}
Los métodos multivariantes pueden clasificarse de la siguiente manera según el caso de estudio:
\begin{itemize}
\item \textbf{Métodos supervisados:} estudian la relación estocástica que hay entre dos conjuntos de variables, un conjunto de variables predictoras y otro de variables respuesta. 
\begin{itemize}
\item \textit{Regresión:} la o las  variables variables respuesta son continuas. 
\item \textit{Clasificación: }la o las variables respuesta son cualitativas. 
\end{itemize}
\item \textbf{Métodos no supervisados: }analizan la estructura y relaciones que hay entre las variables.
\end{itemize}
\end{column}
\end{columns}
\end{frame}

\begin{frame}{Introducción}
\begin{columns}
\begin{column}{0.9\textwidth}
\textbf{Objetivo: }\\
Describir las principales técnicas multivariantes y dar un ejemplo de aplicación sobre datos reales de las mismas.
\end{column}
\end{columns}
\end{frame}


\input{supervisados.tex}
\section{Métodos no supervisados}
\begin{frame}{Métodos supervisados}
\begin{columns}
\begin{column}{0.9\textwidth}

\end{column}
\end{columns}
\end{frame}




\end{document}